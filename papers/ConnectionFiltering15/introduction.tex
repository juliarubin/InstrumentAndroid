%\vspace{-0.05in}
\section{Introduction}
\label{sec:intro} 
%Mobile applications enjoy almost permanent connectivity and the
%ability to exchange information with their own back-end and other
%third-party servers. 
%Recent studies examine that information exchange, showing that applications often release sensitive information, such as the user's precise location, phone number or unique device id~\cite{Enck:Gilbert:Chun:Cox:Jung:McDaniel:Sheth:OSDI10, Egele:Kruegel:Kirda:Vign:NDSS11,Tripp:Rubin:SEC14,Gordon:Kim:Perkins:Gilham:Nguyen:Rinard:NDSS15}. 
%Follow-up works investigate the feasibility of obfuscating or even blocking such information releases altogether~\cite{Hornyack:Han:Jung:Schechter:Wetherall:CCS11}.
%
%Yet, distinguishing between information releases that enable the anticipated application functionality, e.g., getting accurate navigation instructions, and those that are unexpected from the user's point of view, is still a tedious manual task. This paper takes the first steps towards performing such a distinction 
%%identifying and disabling such communication 
%automatically.
%
%In~\cite{Nielsen:Molich:CHI90,Blackmon:Polson:Kitajima:Lewis:CHI02}, the authors claim that  
%"the system should continuously inform the user about what it is doing". 
%Inspired by that principle, we propose to classify communication into 
%\emph{overt} and \emph{covert}: 
%overt communication delivers a tangible value to the user.
%Covert communication is hidden; disabling it leaves the delivered application experience completely intact.   
%
%%Our main insight is that covert : 
%%delivers no value to the user of the application: 
%  
%In practice, covert communication comes with costs such as  
%potential privacy breaches, bandwidth charges, power consumption on the device  
%%and analytic data release, 
%and the unsuspected presence of continued
%communication between the device and remote organizations. 
%In fact, we
%observed that several applications silently spawn services that
%communicate with third-party servers even when the application itself
%is no longer active, with the user completely unaware that the spawned
%services are still running in the background. 
%
%To quantify the extent of the problem,  
%we start by
%analyzing communication patterns of
%%in widely used mobile applications such as ten of the top fifteen most
%popular applications in the Google Play App Store (twitter, WalMart,
%Spotify, Pandora, etc.). Motivated by the significant
%amount of covert communication we found in these applications, 
%we develop a highly precise and scalable static analysis that can identify 
%such communication automatically. We use our analysis to further 
%investigate this unfortunate phenomenon and report on our findings. 
%The following research questions drive this investigation:

Mobile applications enjoy almost permanent connectivity and the
ability to exchange information with their own back-end and other
third-party servers.  This paper shows that much of this communication
does not deliver any tangible value to the application's user:
disabling it leaves the delivered application experience completely
intact.  Yet, this covert communication comes with costs such as
potential privacy breaches, bandwidth charges, power consumption on
the device, and the unsuspected presence of continued communication
between the device and remote organizations. In fact, we observed that
several popular applications, e.g., Walmart and twitter, spawn
services that covertly communicate with remote servers even when the 
application itself is inactive and the user is unaware that the
spawned services are running in the background.


This paper focuses on distinguishing between 
\emph{overt communication} that contributes to the application functionality anticipated by the user, and
\emph{covert communication} that is hidden and unexpected from the user's point of view.
%; disabling it leaves the delivered application experience completely intact.   
%identifying and
%disabling this kind of covert communication. 
We start by
analyzing communication patterns of
%in widely used mobile applications such as ten of the top fifteen most
popular applications in the Google Play app store. Motivated by the significant
amount of covert communication we found in these applications, 
we develop a highly precise and scalable static analysis that can identify 
such communication automatically. We use our analysis to further 
investigate this unfortunate phenomenon and report on our findings. 
The following research questions drive this investigation:

\noindent 
{\bf RQ1: How frequent is covert communication?} % in widely used mobile applications?} 
  % To estimate the significance of the problem, 
We conduct an empirical study that focuses on %identifying and 
investigating the nature of communication in thirteen of the 
top twenty most-popular applications in Google Play (twitter, Walmart,
Spotify, Pandora, etc.).  
%Our study has three major steps. First, we establish baseline
%application behavior.  Towards this end, we record a
%script triggering the application functionality via a series of
%interactions with the application's user interface.  After each
%interaction, the script records the
%application state by capturing a screenshot of the device.
%We then instrument the application to log information
%about all triggered connection statement %, install the instrumented version on a mobile device 
%and execute the instrumented version using the same script.
%Finally, we iterate over all identified triggered connections, disabling each at a time.
%We execute the recorded script again for the version of the application with that connection disabled and compare the obtained screenshots 
%to those of the original run. 
For each connection statement triggered by these applications, we compare the dynamic execution of the original application to that of the application with the corresponding statement being disabled. 

We consider all visual content delivered by the application, including advertisement, as essential and necessary, avoiding any subjective judgment on the relevance of that information to the user.  
That is, we consider application executions as
equivalent if they %result in screenshots that 
differ only in contextual information, such as   
\emph{content} of advertisement information and/or messages in social network
applications, e.g., twitter.
%, and the device's status bar presenting time and battery life measurements. 

%We also
%separately note connections that contribute to presenting
%advertisement content, if the analyzed application has any.
%\vspace{0.05in}
%\noindent\emph{Result Summary:} 
Our study reveals that 63\% of
the \emph{exercised} connection statements are covert~-- disabling
them has no noticeable effect on the observable application
functionality. Interestingly, less than half of such covert connections correspond to communication initiated from known advertisement and analytics (A\&A) packages included in the application.
%; other statements are triggered by the ``core'' application itself. 
Thus, looking at the package information only is not sufficient for distinguishing between
overt and covert communication. 
% Slightly more than 25\% of these correspond to HTTP
%and socket communication. The rest correspond to RPC calls to internal
%services installed on the device: notably, but not exclusively, Google
%advertising and analytics, which further communicate with external
%services.  Moreover, in applications that present advertisement
%material, about 61\% of the connections that do affect the observable
%application behavior are used for advertising purposes only.
%{\bf JR: update.}

\noindent 
{\bf RQ2: Can covert communication be detected statically?}
Detailed investigation of multiple detected cases of covert
connections inspired us to develop a novel static application analysis that
can detect such connections automatically.  The core idea behind our
analysis is to look for cases when both connection success and failure are ``silently'' ignored by the application, i.e., when no information is presented to the user neither on success nor on   failure of the connection. 

For a connection statement, we analyze the portion of the program's
control flow graph that corresponds to the \emph{direct processing} of
the connection.  This includes
\emph{forward stack processing} and \emph{failure handling}
executions.  The former is the non-exceptional execution reachable
after the connection statement but limited to processing of the
Android runtime events that trigger the connection statement.
The latter traverses methods up all possible call stacks from the
connection statement but stops at points at which the
exception raised by the connection statement and all related rethrown
exceptions are cleared.

The direct processing of the connection call is searched for method
call invocations that could target a predefined set of API calls that
affect the user interface.  If such a call is found, the connection
call statement is deemed overt.  Furthermore, if the exception
could be propagated back to the Android runtime (causing the
application to exit), the connection call is deemed overt.
Otherwise, it is deemed covert.

%% For a connection statement, we analyze the portion of the program's
%% control flow graph that corresponds to the \emph{direct processing} of
%% the connection.  This includes (1) the non-exceptional execution
%% reachable after the connection statement and (2) executions up all
%% possible call stacks from the connection statement until the point
%% where the exceptions triggered by the connection statement are
%% cleared.  The direct processing of the connection is searched for
%% method call invocations that affect the user interface.  If such a
%% call is found, the connection call statement is deemed overt.
%% Furthermore, if the exception could be propagated back to the Android
%% runtime (causing the application to exit), the connection call is
%% deemed overt.  Otherwise, it is deemed covert.

%includes  exception is 
%a failure
%handling path includes a method that is an application event handler
%that does not handle the exception, then the application will
%propagate the exception to the Android runtime and could exit on the
%exception, and the connection call is deemed overt.

%Our static analysis is designed to scale to large Android applications
%and to conservatively approximate the behavior of dynamic constructs
%such as reflection and missing semantics such as native methods.  The
%analysis also reasons about application code reachable through Android
%API calls and callbacks by analyzing each application in the context
%of a rich model of the Android
%API~\cite{Gordon:Kim:Perkins:Gilham:Nguyen:Rinard:NDSS15}. 

There are two reasons for the analysis to misclassify a connection as covert. 
First, it does not take into account user interface (UI) updates that occur outside of the connection's direct processing, i.e., %those that result from 
due to the connection altering a local or a remote state. 
Second, the computation of the direct processing itself does not extend beyond the application's byte code, i.e., it does not consider native code and inter-application executions. 
Our experiments show that these cases are rare, which, together with the high scalability achieved by the technique, justifies our design choices.

%
% Such cases might include GUI updates within and application as well as  within nor outside the application not outside of it, e.g., when performing an RPC inter-application communication or when toggling a state of a remote server.
%First, similar to most static analysis techniques for Android, %it does not guarantee soundness because
%it does not consider all runtime semantics, e.g., GUI updates that occur in native code.  

%\emph{stateful} communication patterns with external services: those when a connection 
%alters the state of the server and further communication, that is influenced by the server's state, modifies the user interface. 

%Finally, 

%\vspace{0.05in}
\noindent 
{\bf RQ3: How well does static detection perform?}
To assess the precision and recall of the static analysis technique, we evaluate it on the ``truth set'' established during the in-depth dynamic study described above. %of thirteen popular applications from Google Play.
The results show that our technique features a high precision: 93\% of the covert connection calls identified by the static analysis are indeed classified as covert during the dynamic study. 
Moreover, even though the technique is designed to be conservative and favor high precision over high recall, 
%i.e., it relies on a feasibly analyzable definition of direct processing and conservative call graph construction, 
%specifically w.r.t. reflection, 
it is still able to identify 61\% of all covert connections in the studied applications.

There are only two connections that were misclassified as covert, both due to the reasons mentioned above. The first one occurs because a UI update~-- in this case, presenting icons of additional apps that can be downloaded from Google Play~-- happens outside of the connection's direct processing. The second case, which is responsible for presenting advertisement material via the Google Ads component installed on the device, occurs because the direct processing of the connection does not extend to asynchronous RPC communication with Google services installed on the device. 



To gain further insights on the quality of the technique, we apply it to additional 47 top-popular applications from Google Play. For these applications, we disable all connection statements deemed covert by the analysis. We then employ humans to perform a \emph{usability assessment}: we provide them with two identical devices, one running the original and the other -- the modified version of the application. We ask them to track and report on any observable differences between the run of an application on two devices. 

The results of this  assessment are encouraging: there were no observable differences in 63.8\% of the applications (30 cases). In 25.6\% of the applications (12 cases) differences were related to absence of functionality considered minor by the users, e.g., ads or decorating images. Only 10.6\% of the applications (5 cases) missed functional features considered essential by the users. 
This result implies that our technique produces actionable output that could already be applied to eliminate many cases of covert communication. 


%\vspace{0.05in}
\noindent 
{\bf RQ4: How often does covert communication occur in real-life applications and what are its most common sources?}
%Inspired by the high precision of our technique, 
We apply our technique on the top 500 popular applications from Google Play. This experiment reveals that 46\% of connection statements encoded in these applications 
are deemed covert.
Most common sources of covert communication are Google services %for mobile developers 
and various A\&A services. Yet, these are not the exclusive source of covert communication, and not all communication made from these packages is covert. 
%We conjecture 
%that applications commonly register for various such services without eventually using them. 
%Additional common targets are advertisement, analytics and gaming services. 

%%\vspace{0.05in}
%\noindent 
%{\bf Significance of the Work.}  Our work focuses on benign
%applications that can be downloaded from popular application stores
%and that are installed by millions of users.  By identifying and
%highlighting application functionality hidden from the user, the goal
%is to encourage application developers to produce more transparent and
%trustworthy applications. 
%{\bf JR: say that ads are in and just disabling ads in not a solution. Do not blame developers.} 
%The identification of potential privacy
%violations in previous versions of popular Android
%applications~\cite{Enck:Gilbert:Chun:Cox:Jung:McDaniel:Sheth:OSDI10,Egele:Kruegel:Kirda:Vign:NDSS11,Tripp:Rubin:SEC14} followed by the
%elimination of these violations in current Android applications
%provides encouraging evidence that such an improvement is feasible.
%

\noindent 
{\bf Significance of the work.}  
In~\cite{Nielsen:Molich:CHI90,Blackmon:Polson:Kitajima:Lewis:CHI02,Ko:Zhang:CHI11}, the authors state that  
``the system should continuously inform the user about what it is doing''.
Inspired by this principle, the goal of our work is to identify 
%and highlight 
application functionality hidden from the user.
This paper studies the extent of that problem in benign
applications downloaded from popular application stores
and installed by millions of users.
We believe that our findings help focus the effort of improving
transparency and trust in the mobile application domain. 


%{\bf JR: say somewhere that free apps have more ads. Say something about the business model.}

%\vspace{0.05in}
\noindent 
{\bf Contributions.}
The paper makes the following contributions:

\vspace{-0.05in}
\begin{enumerate}[leftmargin=0.5cm]\setlength{\itemsep}{-0.01in}

\item It sets \emph{a new problem} of distinguishing between overt and covert communication in an automated manner.
% with the goal of improving
%transparency and trustworthiness of mobile applications.

%This problem is complementary and orthogonal to the problem of information leakage detection. 

\item It proposes \emph{a semi-automated dynamic approach} for detecting covert communication in Android applications. 
%s that does not require access to the application source code. 
The approach relies on interactive injection of connection failures
in application binaries and identification of cases in which such injections do not
affect the observable application functionality.

\item It provides \emph{empirical evidence} for the prevalence of covert connections in real-life applications. Specifically, it shows that 63\% of the connections attempted by thirteen top-popular free applications on Google Play fall into that category.    

\item It proposes \emph{a static technique} that operates on
application binaries and identifies covert connections. 
% -- those
%that do not directly propagate any effects back to the application's user. 
The
technique is highly scalable and precise: out of 47 highly popular
applications on Google Play, 63.8\% worked without any interference
and further 25.6\% worked with only insignificant interference when
disabling all covert connections identified by the technique.
The technique also features 93\% precision and 61\% recall when evaluated on the ``truth set'' established dynamically.  
% precision and recall of the technique is 93\% and 61\%, respectively, when evaluated against the empirically established truth set.       

\item It provides \emph{quantitative evidence} for the prevalence of covert connections in the 500 top-popular free applications on Google Play and identifies their common patterns. 

%, showing that 84\% of connections encoded in these applications can be deemed as covert.

\end{enumerate}

%% The remainder of the paper is structured as follows. Section~\ref{sec:study} describes the empirical study we conducted for gaining insight into the nature of information releases in mobile applications. Section~\ref{sec:analysis} presents the static analysis technique designed for identifying covert information releases. 
%% Section~\ref{sec:evaluation} discusses results of its evaluation on real-live examples. Section~\ref{sec:limitations} discusses the limitations and threats to validity of our work. Section~\ref{sec:related} presents the related work, while Section~\ref{sec:conclusions} concludes. 



