\section{Failure-Handling Analysis}
\label{sec:analysis}

In this section we describe the static analysis algorithm we employ to
automatically classify connections.  Given an Android application, the
static analysis classifies each statement that may invoke a
connection call as either {\it essential} or {\it non-essential}.
Intuitively, we define an essential connection statement, $s$, as
meeting either of the following criteria:

\begin{enumerate}
\item{\bf User-Interface Cue}: When $s$ triggers an error, the user
may be notified of the error via a user interface cue during error
handling.
\item {\bf Program Exit}: When $s$ triggers an error, the program 
   may stop executing.  
\end{enumerate}

\noindent Conversely, a non-essential connection call does not meet
either of the two criteria.  

Android applications are developed in Java, and program execution
follows the semantics of Java. In an Android application, each
connection call $s$ may generate an exception (of dynamic type $e$)
that reaches a subset of the program's \lstinline!catch! blocks.  At
runtime, when $e$ is triggered by $s$, the executing method's trap
table is consulted, and if no \lstinline!catch! blocks are defined to
handle $e$ at $s$, then $e$ is passed back up the stack to the
calling method at the calling statement, and the process repeats.  If
the Android runtime is returned to during the stack unwinding, the
application is typically exited with an error.  


\begin{figure}
\begin{lstlisting}[numbers=left, escapeinside={(*@}{@*)},basicstyle=\ttfamily\scriptsize]
public class ApplicationClass {
    void f() {
        try {
            g();
            (*@{\it stmtA};@*)
        } catch (AdvertisingException e) {
            (*@{\it stmtB};@*)
        }
        (*@{\it stmtC};@*)
    }
}
public class AdvertisingAPIClass {
    void g() throws AdvertisingException {
        try {
            (*@{\it stmtD};@*)
            connect();  //throws RemoteException (*@\label{connect-fig-line}@*)
            (*@{\it stmtE};@*)
        } catch (RemoteException e) {
            (*@{\it stmtF};@*)
            throw new AdvertisingException();            
        }        
        (*@{\it stmtG};@*)
    }
}
\end{lstlisting}
\caption{\label{fig:failure-handling}An example of failure handling of
  of a connection call at line ~\ref{connect-fig-line}.  Statements
  {\it stmtF}, \lstinline!throw new AdvertisingException()!, and {\it stmtB}
  are in the failure handling of this statement for exception \lstinline!RemoteException!.
}
\end{figure}

\begin{description}[leftmargin=0cm,listparindent=0pt,itemindent=0cm]
\item \textsc{\bfseries{Definition (Rethrown Exception)}}.  A
rethrown exception occurs when a \lstinline!catch! block catches an
exception, but before the block is exited, a statement reachable from
the block explicitly throws the same exception object, or throws a new
exception object.  The process of searching the stack for a handler
begins anew.

\item \textsc{\bfseries{Definition (Failure Handling)}}. The failure
  handling of a connection call $s$ for exception type $e$ is defined
  as the execution path that starts when an exception of type $e$
  propagates to connection call $s$ and ends when the {\it last}
  \lstinline!catch! block is exited that handles $e$ or a rethrown
  exceptions of $e$.
\end{description}

Intuitively, the failure handling of $s$ on $e$ is the computation
that handles $e$ and any failure triggered by the handling of $e$
(through rethrown exceptions).  Failure handling is finished when all
exceptions triggered by $e$ are handled and flow returns to normal
execution.  

Figure~\ref{fig:failure-handling} give a simplified representation of 
failure handling pattern that we observed in the twitter application. 
Method \lstinline!f! invokes method
\lstinline!g!.  In \lstinline!g!, a connection call is encountered on
line~\ref{connect-fig-line}; assume during execution this connection call
throws a \lstinline!RemoteException!.  The failure handling for this
combination of connection call and exception is the set of statements:
{\it stmtF}, \lstinline!throw new AdvertisingException()!, and {\it stmtB}.
These statements are executed from the start of the handling of thrown
\lstinline!RemoteException! to the end of the handling of the rethrown
\lstinline!AdvertisingException!.

\begin{description}[leftmargin=0cm,listparindent=0pt,itemindent=0cm]
\item \textsc{\bfseries{Problem}}: Analyze each connection call,
$s$, in an Android application to determine whether the application
could possibly exit on a failure at $s$ or could modify the user
interface during a failure handling path of $s$.
\end{description}

To solve this problem statically, our failure-handling analysis
conservatively calculates all possible failure handling for exceptions
that denotes connection failure for each connection call in the
application.  If there exists a failure-handling path of $s$ on $e$
that may include a call to a method that notifies the user of the
failure, then $s$ is considered essential.  If it is possible for $e$,
when triggered at $s$, to propagate back up the stack to the Android
runtime, $s$ is considered essential.

\subsection{Static Analysis of Android Applications}

%% Android applications are developed in Java and compiled to Dalvik
%% executable format (dex) byte code.  Android applications are
%% distributed as packages that do not include the Java source; packages
%% include only the dex byte code and application resources such as
%% images, sounds, and GUI declarations.  
Static analysis of Android
applications is notoriously difficult because of issues
including~\cite{Gordon:Kim:Perkins:Gilham:Nguyen:Rinard:NDSS15}:

\begin{itemize}[leftmargin=0.5cm]

\item Android applications execute in the context of the Android API
  and runtime.  The application thus represents an incomplete program.
  
\item The Android API and runtime comprises multiple millions of lines of
  code implemented in multiple programming languages.  Furthermore,
  much of the implementation is left for device manufactures to
  implement, and is thus proprietary and closed-source. 

\item Applications are event-driven and dynamic by nature.
  Applications define event handlers for possible runtime events that
  are triggered in the Android runtime, and passed to the application
  for handling. 

\item Applications interact heavily with the Android API.  The Android
  API includes most of the Java standard library, plus additional
  utility and resource access classes.

\item Android application packages typically ship with third-party
  libraries for performing operations such as advertising, analytics,
  and interaction with remote services.  These libraries are commonly
  large, obfuscated, and include heavy use of reflection.

\end{itemize}

It is not feasible for a static analysis to include analysis of the
source code of the Android runtime and API because of the size and
multi-language nature of this code base.  Thus static analysis must
either model the Android application execution environment, or account
for possible dynamic program behaviors with conservative analysis
choices; otherwise some runtime behaviors could be unconsidered.
Precise, whole-program analysis runs the high-risk of missing dynamic
program behavior and not scaling to real-world Android
application~\cite{Gordon:Kim:Perkins:Gilham:Nguyen:Rinard:NDSS15}.

Our analysis employs a class hierarchy analysis (CHA)~\cite{Dean1995}
to build a call graph with refinement achieved by intra-procedural
data-flow analysis.  After much experimentation with higher precision,
though brittle, points-to analysis techniques, this analysis
combination gave us the best performance for the classification task.
We augment the call graph to account for reflected method calls, and
conservatively account for exceptions that can be thrown by native
code.  Our failure-handling analysis over-approximates the runtime
behaviors of the applications, and under-approximates the connection
calls that could be non-essential.

The presented analysis has the following limitations. Dynamically
loaded code is not be considered.  The analysis considers only checked
exceptions.  A best-effort, though aggressive, policy is used to
account for reflection semantics; this policy could miss possible
runtime semantics.

Our analysis is implemented in the Soot Java Analysis
Framework~\cite{Vallee-Rai2000} and utilizes libraries and the Android
API model provided by
DroidSafe~\cite{Gordon:Kim:Perkins:Gilham:Nguyen:Rinard:NDSS15}. The
presentation of the analysis below assumes the application is
represented in the Jimple intermediate language~\cite{Vallee-Rai2000}.

\subsection{Call Graph Construction}

%%  A call graph represents the possible dynamic calling
%% relationship between methods of the application.  Each node represents
%% a method, and each edge, $(f,g)$ indicates that $f$ may call $g$ at
%% run time.

Our algorithm first computes a static call graph based on CHA
analysis. To compute a call graph, we augment the application code with the
DroidSafe Android Device Implementation
(ADI)~\cite{Gordon:Kim:Perkins:Gilham:Nguyen:Rinard:NDSS15}.  The ADI
is a Java-based model of the Android runtime and API that attempts to
present full runtime semantics for commonly-used classes of the
runtime and API.  Our call graph construction does not traverse into
Android API methods.  However, we found it necessary to account for
API calls that immediately jump back into the application.  For
example, if an application method, $m$ calls
\lstinline!Thread.start()! on a receiver that is an application class,
$t$, we found it necessary to add the edge to the callgraph ($m$,
$t$.\lstinline!run()!).  This includes the started thread $t$ in
failure handler if $m$ is encountered.

To achieve this in general, we add to the call graph edges of the type
($m$, $n$) where there is an edge ($m$, \lstinline!api-method!), the
call of \lstinline!api-method! is passed a value that is a reference
to an application class, and \lstinline!api-method! calls method $n$
on the passed application class value.  This strategy adds to our
callgraph the edges for the \lstinline!Thread.start()! to the
\lstinline!Thread.run()! discussed above.

Furthermore, the call graph is augmented to account for reflected
method calls in the application using the following policy.  When a
reflected call is found, we add edges to the graph that target all
methods of the same package domain as the caller (e.g.,
\lstinline!com.google!, \lstinline!com.facebook!).  The edges are pruned by the
following strategy: if the number of arguments and argument types to
the call can be determined using a def-use analysis~\cite{Aho2006},
then we limit the edges to only targets that have the same number and
types of arguments.  This strategy works well for us in
practice and aggressively accounts for reflection semantics.


\subsection{Failure Handler Analysis}

We organize the static failure-handling analysis as a recursive
traversal on the call graph for ease of understanding.  An iterator
over all application statements calls the analysis separately for the
combination of each statement in the application that could target a
connection call and an exception that indicates communication failure.
Table~\ref{tbl:connections} lists the target methods that we consider
connection calls and each method's associated failure exception.

%resize line numbers
\algrenewcommand\alglinenumber[1]{\scriptsize #1:}
\begin{figure}
\scriptsize
\begin{algorithmic}[1]
\algblockdefx[HEADER]{inputdef}{outputdef}[1]%
{\textbf{Input}: #1}[1]{\textbf{Output}: #1}
\Procedure{FindCatches}{meth, stmt, ex, visiting, stack, cg}

\If {(stmt, ex) $\in$ visiting || (stmt, ex) $\in$ essential}
    \State \textbf{return}
\EndIf

\State visiting $\gets$ visiting $\cup$ (stmt, ex)

\State catchBlockStart $\gets$ \textsc{FindCompatCatch}(meth, stmt, ex)

\If {catchBlockStart $=$ \textbf{null} }
\label{alg:no-local-hander-start}
\If {\textsc{IsEventHandler}(meth)}
\label{alg:event-handler-line}
\State essential $\gets$ essential $\cup$ (stmt, ex)
\State \textbf{return}
\EndIf

\For { (predStmt, predMeth) $\in$ \textsc{GetPreds}(cg, meth)}
\If {stack $\neq \emptyset$ \textbf{and} (predStmt, predMeth) $\neq$
  \textsc{Peek}(stack)} 
\label{alg:stack-test-line}
\State \textbf{continue}
\EndIf 

\State newStack $\gets$ stack
\State \textsc{Pop}(newStack)
\label{alg:stack-pop-line}
\State \textsc{FindCatches}(predMeth, predStmt, ex, 
\State         \hspace{2cm}visiting, newStack, cg)

\If {(predStmt, ex) $\in$ essential}
\State essential $\gets$ essential $\cup$ (stmt, ex)
\State \textbf{return}
\EndIf
\label{alg:no-local-hander-end}
\EndFor

\Else 
\State catchStmts $\gets$ \textsc{GetCatchStmts}(catchBlockStart, meth)
\label{findcatches-analyze-call}
\State \textsc{AnalyzeHandling}(meth, stmt, catchStmts,
\State \hspace{2cm}visiting, $\emptyset$, stack, cg)
\EndIf

\EndProcedure
\end{algorithmic}
\caption{Find \lstinline!catch! blocks for exception thrown at statement.}\label{alg:findcatches}
\end{figure}

The analysis starts with the \textsc{FindCatches} procedure listed in
Figure~\ref{alg:findcatches}.  For each start of the analysis on a
statement and exception pair, $s$ and $e$, respectively, the procedure
first consults $s$'s containing method to find an appropriate
\lstinline!catch!; if $e$ is not caught locally, the analysis
recursively visits all direct predecessors of the method to find
\lstinline!catch! blocks that trap the call statement edge (lines
\ref{alg:no-local-hander-start}-\ref{alg:no-local-hander-end}).  For
  each predecessor, $p$, if a catch is not found that wraps the call
  edge, then $p$'s direct predecessors are visited, and so on.

\begin{figure}[t]
\begin{algorithmic}[1]
\scriptsize
\algblockdefx[HEADER]{inputdef}{outputdef}[1]%
{\textbf{Input}: #1}[1]{\textbf{Output}: #1}
\Procedure{AnalyzeHandler}{meth, exceptStmt, stmts, 
  visiting, handledStmts, stack, cg}
\If {stmts $\in$ handledStmts}
\State \textbf{return}
\EndIf
\State handledStmts $\gets$ handledStmts $\cup$ stmts

\For {each stmt $\in$ stmts}
\If {\textsc{HasInvoke}(stmt)}
\For {(succStmt, succMeth) $\in$ \textsc{GetSuccs}(cg, stmt)}

\If {\textsc{IsUIMethod}(succMeth)}
\label{alg:ui-method-found}
\State essential $\gets$ essential $\cup$ exceptStmt
\State \textbf{return}

\ElsIf {\textsc{IsNativeMethod}(succMeth)}
\label{alg:native-method-line}
\For {nativeEx $\in$ \textsc{GetThrowsExceptions}(succMethd)}
\State \textsc{FindCatches}(meth, stmt, nativeEx, visiting, stack, cg)
\EndFor

\Else
\label{alg:ah-app-method-call-start}
\State newStack $\gets$ stack
\State \textsc{Push}(newStack, (succStmt, succMeth))
\label{alg-push-stack-lines}
\State succStmts $\gets$ \textsc{GetBodyStmts}(succMeth)
\State \textsc{AnalyzeHandler}(succMeth, exceptStmt, succStmts,
\State \hspace{2cm}visiting, handledStmts, newStack, cg)
\label{alg:ah-app-method-call-end}
\EndIf 

\EndFor

\ElsIf {\textsc{IsThrowStmt}(stmt)}

\State rethrownTypes = $\emptyset$
\label{alg:found-throws-start}
\For {defStmt $\in$ \textsc{GetLocalDefs}(\textsc{GetOp}(stmt))}

\If {\textsc{IsAlloc}(defStmt)}
\State rethrownTypes $\gets$ rethrownTypes $\cup$ 
\State \hspace{2.5cm} \textsc{GetAllocType}(defStmt)
\ElsIf {\textsc{IsCaughtExceptionStmt}(defStmt)}
\State rethrownTypes $\gets$ rethrownTypes $\cup$ 
\State \hspace{0.5cm} \textsc{GetPossibleThrownTypes}(meth, defStmt)
\label{alg:getpossiblethrowntypes}
\Else
\label{alg:unknown-last-def-throw}
\State essential $\gets$ essential $\cup$ exceptStmt
\State \textbf{return}
\EndIf

\EndFor \label{alg:calc-rethrow-types-end}

\For {rethrownType $\in$ rethrownTypes}
\label{alg:findcatches-rethrown-line}
\State \textsc{FindCatches}(meth, stmt, rethrownType, visiting, stack,
cg)

\If {stmt $\in$ essential}
\label{alg:progagate-line}
\State essential $\gets$ essential $\cup$ exceptStmt
\State \textbf{return}
\EndIf
\EndFor
\label{alg:found-throws-end}
\EndIf

\EndFor

\EndProcedure
\end{algorithmic}
\caption{Analyze reachable statements during handling for UI
  interaction or rethrown exceptions.}\label{alg:analyzehandler}
\end{figure}

For each \lstinline!catch! that is found during the
\textsc{FindCatch}, handler analysis of the reachable statements of
the \lstinline!catch! block is performed (\textsc{FindCatch}, line
\ref{findcatches-analyze-call}).  Figure~\ref{alg:analyzehandler}
gives the listing of the handler analysis procedure
\textsc{AnalzyeHandler}. The analysis considers the reachable
statements inter-procedurally and flow-insens\-itively.  Handler
analysis searches for: (1) calls to application methods, (2)
\lstinline!throw! statements (3) calls to native methods, and (4)
possible calls to UI methods. When the analysis finds a call to an
application method, it pushes the current statement and method onto
the stack and recursive calls itself for the new method to analyze the
new method's statements (lines
\ref{alg:ah-app-method-call-start}-\ref{alg:ah-app-method-call-end}).
If analysis finds a \lstinline!throw!  statement, the handler analysis
spawns a new \textsc{FindCatches} analysis to find all the
possible handlers of each rethrown exception (lines
\ref{alg:found-throws-start}-\ref{alg:found-throws-end}).  If analysis
finds a call to a native method, we assume that it will throw all
exceptions it is defined to throw, handler analysis spawns a
\textsc{FindCatches} instance for each exception declared throws
(line~\ref{alg:native-method-line}).  If a call is encountered that
could target a UI method, then the statement that began the handler
analysis is considered essential since the error handling affects the
user interface (line~\ref{alg:ui-method-found}).

\textsc{FindCatches} and \textsc{AnalyzeHandler} maintain a set of
statement and exception pairs, essential, that records pairs that are
calculated as essential.  After all connection call statement and
exception pairs are analyzed, pairs not in the essential set are
considered non-essential.

\lstMakeShortInline[basicstyle=\scriptsize\ttfamily,keywordstyle=\color{DarkPurple},breaklines=true]+

\begin{table}[t]
\small
\renewcommand*{\arraystretch}{1.3}
\caption{Considered UI Elements.}
\label{tbl:ui}
\centering
\tabcolsep=1.5pt
%\resizebox{\columnwidth}{!}{%
\begin{tabular}{|l|P{3.7cm}|P{4cm}|}
\hline
& \textbf{Class or Interface} & \textbf{Methods} \\
\hline
1. & +android.app.Dialog+                                 & +setContentView+ \\
2. & +android.support.v7.app.ActionBarActivityDelegate+  & +setContentView+ \\
3. & +android.view.View+                                  & +onLayout+, +layout+, +onDraw+, +onAttachedToWindow+ \\
4. & +android.view.ViewGroup+                             & +addView+, +addFocusables+, +addTouchables+, +addChildrenForAccessibility+ \\
5. & +android.view.ViewManager+                           & +addView+, +updateViewLayout+ \\
6. & +android.view.WindowManagerImpl.CompatModeWrapper+  & +addView+ \\
7. & +android.webkit.WebView+                             & +loadData+, +loadDataWithBaseURL+, +loadUrl+ \\
8. & +android.widget.TextView+       & +append+, +setText+ \\
9. & +android.widget.Toast+        & +makeText+ \\
\hline
\end{tabular}
%}%resizebox
\end{table}

\lstDeleteShortInline+


The set of target methods that are considered as affecting the user
interface are listed in Table~\ref{tbl:ui}.  We also define all
overriding methods of the methods listed in the table as UI methods.

A stack of pairs of method call statement and method is maintained
during the analysis.  The analysis uses the stack to focus the handler
search in \textsc{FindCatches} after a method call has been performed
by a handler further up the stack. When we initiate the analysis for
a connection call, the stack is empty and the analysis in
\textsc{FindCatches} has to search all possible stacks (predecessor of
the containing method) for handlers of the connection statement's
exception.  However, once a handler is found, and the handler calls a
sequence of methods that ends in a possible rethrown exception, the
sequence of methods defines the only stack that should be searched for
a handler of the rethrown exception.  The stack is pushed on line
\ref{alg-push-stack-lines} of \textsc{AnalyzeHandler} for each method
call of a reachable handler code.  During the handler search of the
execution stack in \textsc{FindCatches}, the
stack is consulted to guide the search on line
\ref{alg:stack-test-line}, only visiting the edge is at the head of
the stack. The stack is popped when visiting a caller method of the
current method in \textsc{FindCatches} line~\ref{alg:stack-pop-line}.

During handler search in \textsc{FindCatches}, if no handler is found
locally, and the method is a possible entry point called from the
Android runtime, then we conservatively calculate that the exception
and excepting statement could cause application exit, so the pair is
added to the essential set (line~\ref{alg:event-handler-line} of
\textsc{FindCatches}). 

During handler analysis, if a \lstinline!throw! statement is
encountered in reachable code of a handler, the analysis needs to
determine the possible type of the thrown value, and then start a new
search for the handler.  The \textsc{AnalyzeHandler} procedure
calculates local def-use chains for the method it is analyzing.  It
uses the local def-use information to calculate the types of the
exception.  In lines ~\ref{alg:found-throws-start} through
\ref{alg:calc-rethrow-types-end} the analysis considers all local
reaching definitions of the thrown value.  If an allocation statement
reaches, then add the allocated types to the possible types of
rethrown exceptions. If a caught exception statement\footnote{A caught
  exception statement is a statement that defines that start of a
  \lstinline!catch! block and assigns a local variable to the
  exception object caught by the block.}, $c$, reaches the
\lstinline!throw! statement, then the \lstinline!try! block associated
with \lstinline!catch! block of $c$ is analyzed for all checked
exceptions that could be thrown.  This is performed in
\textsc{GetPossibleThrownTypes} call on line
\ref{alg:getpossiblethrowntypes} of \textsc{AnalyzeHandler}.

If any other type of statement is a definition that reaches the thrown
value, then the analysis cannot determine the exception type and the
connection call (or rethrown exception statement) is considered
essential (line \ref{alg:unknown-last-def-throw}).  If only
allocations and caught exception statements reach the thrown value,
then the handler analysis spawns a new \textsc{FindCatches} instance
to analyze the failure handling.  The classification of the thrown
statement is propagated to the current excepting statement in
line \label{alg:progagate-line} of \textsc{AnalyzeHandler}.

\begin{figure}[t]
\begin{algorithmic}[1]
\scriptsize
\Procedure{GetPossibleThrownTypes}{meth,stmt}
\State thrownTypes = $[]$
\State tryStmts = \textsc{GetTryBlock}(meth,stmt)

\For {tryStmt $\in$ tryStmts}

\If {\textsc{IsThrowStmt}(tryStmt)}

\For {defStmt $\in$ \textsc{GetLocalDefs}(\textsc{GetOp}(stmt))}
\If {\textsc{IsAlloc}(defStmt)}
\State rethrownTypes $\gets$ rethrownTypes $\cup$ 
\State \hspace{1cm} \textsc{GetAllocType}(defStmt)
\ElsIf {\textsc{IsCaughtExceptionStmt}(defStmt)}
\State reThrownTypes $\gets$ rethrownTypes $\cup$
\State \hspace{1cm} \textsc{GetPossibleThrownTypes}(meth, defStmt)
\Else 
\State \textbf{return} $\emptyset$
\label{alg:getpossiblethrown-return-null}
\EndIf
\EndFor

\ElsIf {\textsc{HasInvoke}(tryStmt)}

\For {(succStmt, succMeth) $\in$ \textsc{GetSuccs}(cg, tryStmt)}
\State thrownTypes $\gets$ thrownTypes $\cup$ 
\State \hspace{1cm}\textsc{GetThrowsExceptions}(succMeth)
\EndFor

\EndIf

\EndFor

\State \textbf{return} thrownTypes

\EndProcedure
\end{algorithmic}
\caption{For an exception statement $stmt$ and $stmt$'s containing
  method, return the possible exception types that flow to
  $stmt$}\label{alg:GetPossibleThrownTypes}
\end{figure}

Figure~\ref{alg:GetPossibleThrownTypes} gives the algorithm for
\textsc{GetPossibleThrownTypes}.  First, the method calculates the
\lstinline!try! block that associates with the \lstinline!catch! block
that encloses $stmt$. Next, the procedure examines \lstinline!throw!
and call statements of the \lstinline!try!.  For a call statement, the
procedure adds to the return list all exception types declared throws
by all methods that the call can target.  For a \lstinline!throw!
statement, the reaching definitions of the thrown value are
calculated.  If the reaching definition is an allocation, then add to
the return list the type of the allocation.  If the reaching
definition is a caught exception statement, then
\textsc{GetPossibleThrownExceptions} recursively calls itself to find
the nesting try block statements and continue the calculation.  If a
definition of any other statement type can reach the thrown value,
then the procedure returns null to denote that it cannot calculate the
thrown type (line~\ref{alg:getpossiblethrown-return-null}).  This
situation causes the examined connection call or rethrown statement in
\textsc{AnalyzeHandler} to be labeled essential (we have not included
the code in the algorithms to propagate this situation to the
essential set, though it is in our implementation).



\subsection{Helper Procedures}

The analysis employs the following helper procedures: 

\begin{description}[leftmargin=0cm,listparindent=0pt,itemindent=0cm]

\item \textsc{FindCompatCatch}($meth$,$stmt$,$ex$): Return the first
  statement of the \lstinline!catch! block that will handle an
  exception of type $ex$ thrown at statement stmt in method meth.

\item \textsc{IsEventHandler}($meth$): Return true if method $meth$
  overrides a method defined in the Android API.  This method
  over-approximates the methods that can be called by the Android
  runtime to handle events.

\item \textsc{GetCatchStmts}($stmt$,$meth$): Given the start of a
  \lstinline!catch! block defined in the trap table of method $meth$,
  return all statements that were defined in the source code for the
  \lstinline!catch! block of $stmt$.  Since the dex bytecode does not
  provide the ending statement of traps, we need to calculate the
  extend of the catch block.  \textsc{GetCatchStmts} takes advantage
  of the property that Java compilers do not produce code that jumps
  from outside of the catch block into the middle of a catch block.
  So to calculate the \lstinline!catch! block's extent,
  \textsc{GetCatchStmts} (1) produces a control flow graph (CFG) for
  $meth$, (2) colors all statements reachable from $stmt$ in the CFG,
  (3) for each statement, $c$ of $meth$, if all predecessors of $c$ in
  the CFG are colored then $c$ is included in the set of statements
  that are returned ($stmt$ is also included in the return set). This
  method calculates an over-estimation of \lstinline!catch! block
  extents, e.g., it includes \lstinline!finally! blocks.

\item \textsc{GetTryBlock}($meth$,$stmt$): Given a statement $stmt$
  that begins a \lstinline!catch! block in method $meth$, return the
  list of statement of try block associated with the enclosing
  \lstinline!catch! block of $stmt$.

\end{description}

