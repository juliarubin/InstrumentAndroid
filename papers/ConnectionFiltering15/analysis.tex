%\vspace{-0.05in}
\section{Static Analysis for Classifying Connections}
\label{sec:analysis}

\lstMakeShortInline[basicstyle=\scriptsize\ttfamily,keywordstyle=\color{DarkPurple},breaklines=false]+

In this section we describe the static analysis algorithm we employ to
automatically classify connections.  Android applications are
developed in Java as a series of external event handler routines,
e.g., button click and application exit.  Given an Android
application, the static analysis classifies each statement that may
invoke a connection call as either {\it overt} or {\it covert}.  We
begin by giving a precise definition of overt and covert connections
assumed by the analysis.

%% In an Android application, each
%% connection call $s$ may generate an exception (of dynamic type $e$)
%% that reaches a subset of the program's \lstinline!catch! blocks.  At
%% runtime, when $e$ is triggered by $s$, the executing method's trap
%% table is consulted, and if no \lstinline!catch! blocks are defined to
%% handle $e$ at $s$, then $e$ is passed back up the stack to the
%% calling method at the calling statement, and the process repeats.  If
%% the Android runtime is returned to during the stack unwinding, the
%% application is typically exited with an error.  

%% \begin{}
%% \begin{lstlisting}[numbers=left, escapeinside={(*@}{@*)},basicstyle=\ttfamily\scriptsize]
%% public class ApplicationClass {
%%     void f() {
%%         try {
%%             g();
%%             (*@{\it stmtA};@*)
%%         } catch (AdvertisingException e) {
%%             (*@{\it stmtB};@*)
%%         }
%%         (*@{\it stmtC};@*)
%%     }
%% }
%% public class AdvertisingAPIClass {
%%     void g() throws AdvertisingException {
%%         try {
%%             (*@{\it stmtD};@*)
%%             connect();  //throws RemoteException (*@\label{connect-fig-line}@*)
%%             (*@{\it stmtE};@*)
%%         } catch (RemoteException e) {
%%             (*@{\it stmtF};@*)
%%             throw new AdvertisingException();            
%%         }        
%%         (*@{\it stmtG};@*)
%%     }
%% }
%% \end{lstlisting}
%% \vspace{-0.2in}
%% \caption{\label{fig:failure-handling}An example of failure handling.
%% % of
%% %  of a connection call at line ~\ref{connect-fig-line}.  Statements
%% %  {\it stmtF}, \lstinline!throw new! \lstinline!AdvertisingException()!, and {\it stmtB}
%% %  are in the failure handling of this statement for exception \lstinline!RemoteException!.
%% }
%% \vspace{-0.1in}
%% \end{figure}

%% \begin{description}[leftmargin=0cm,listparindent=0pt,itemindent=0cm]

%% \item \textsc{\bfseries{Definition (Rethrown Exception)}}.  A rethrown
%%   exception occurs when a \lstinline!catch! block catches an
%%   exception, but before the block is exited, a reachable statement
%%   throws a new or the same exception.
  
%% \item \textsc{\bfseries{Definition (Failure Handling)}}. The failure
%%   handling of a connection call $s$ for exception type $e$ is defined
%%   as all the execution paths that 

%%   start at all catch blocks that handle exceptions of type $e$ at $s$
  
%%   and 

%%   end when (1) 

%%  end when a \lstinline!catch!
%%   block is exited such that $e$ and all exceptions that could be
%%   rethrown on the path from $s$ are caught.


%% \end{description}


%% Figure~\ref{fig:failure-handling} give a simplified representation of
%% failure handling pattern that we observed in the twitter application.
%% Method \lstinline!f! invokes method \lstinline!g!.  In \lstinline!g!,
%% a connection call is encountered on line~\ref{connect-fig-line};
%% assume during execution this connection call throws a
%% +RemoteException+.  The failure handling for this
%% connection call and exception is the set of statements:
%% {\it stmtF}, \lstinline!throw new! +AdvertisingException()+,
%% and {\it stmtB}.  These statements are executed from the start of the
%% handling of thrown \lstinline!RemoteException! to the end of the
%% handling of the rethrown \lstinline!AdvertisingException!.



\begin{description}[leftmargin=0cm,listparindent=0pt,itemindent=0cm]
\item \textsc{\bfseries{Definition (Overt and Covert
    Connections)}}.  For connection statement $s$, $s$ is classified as
   {\it overt} if it meets at least one of the following
  criteria:
  
\begin{enumerate}
\item  {\bf UI Cue on Failure}: When $s$ triggers an
  exception $e$,
  the user may be notified of the failure via a user interface cue
  during {\it failure handling} of $s$ on $e$.


\item {\bf Program Exit on Failure}: When $s$ triggers an exception $e$, the
program may stop executing due to an execution path that propagates
$e$ back up to the Android runtime.

\item {\bf UI Cue on Success}: When $s$ succeeds, there
  exists a possible modification to the user interface {\it before}
  either (a) the next external event is handled by the application or
  (b) processing continues beyond the {\it failure handling} of $s$.

\end{enumerate}

\noindent Conversely, a covert connection call does not meet any of
the criteria.  

\end{description}

Failure handling is defined precisely in
Section~\ref{sec:failure-handling}.  Intuitively, the failure handling
of connection statement $s$ on exception type $e$ are the computation
paths that handle $e$ and any failures triggered by the handling of
$e$ (through rethrown exceptions).  Failure handling is finished when
all exceptions triggered by $e$ are handled and flow returns to normal
execution.  

In what follows, we discuss how we compute a static call graph
that represents possible method targets of each call expression in the
Android application. We then discuss our analysis on top of this graph. 
For each connection statement, the analysis consists of three steps:
failure handler analysis, success forward analysis,
and success backward analysis. 
% The analysis terminates if a
%step determines that the connection is overt. Otherwise, a
%connection is considered covert.

Table~\ref{tbl:connections} lists the target methods that are considered as
connection calls.  The set of target methods that are considered as
affecting the user interface are listed in Table~\ref{tbl:ui} (also
included are all overriding methods).



%% \begin{description}[leftmargin=0cm,listparindent=0pt,itemindent=0cm]
%% \item \textsc{\bfseries{Problem}}: For each connection call in an
%%   Android application, determine whether it is overt or
%%   covert.
%% \end{description}

%% To solve this problem statically, our failure-handling analysis
%% conservatively calculates all possible failure handling for exceptions
%% that denotes connection failure for each connection call in the
%% application.  If there exists a failure-handling path of $s$ on $e$
%% that may include a call to a method that notifies the user of the
%% failure, then $s$ is considered overt.  If it is possible $e$,
%% when triggered at $s$, to propagate back up the stack to the Android
%% runtime, $s$ is considered overt.


%% Android applications are developed in Java and compiled to Dalvik
%% executable format (dex) byte code.  Android applications are
%% distributed as packages that do not include the Java source; packages
%% include only the dex byte code and application resources such as
%% images, sounds, and GUI declarations.  

\subsection{Constructing an Accurate Call Graph for Android Apps }

Static analysis of Android applications is notoriously difficult
because of the complexity and dynamic nature of the Android runtime
and API; precise, whole-program analysis runs the high-risk of missing
dynamic program behavior and not scaling to real-world Android
applications.  Static analysis must either model the Android execution
environment or account for possible dynamic program behaviors with
conservative analysis choices; otherwise some runtime behaviors could
be unconsidered.

Our analysis over-approximates the runtime behaviors of the
applications, and under-approximates the connection calls that could
be covert.  The analysis prioritizes high precision over high recall
because we do not want to block connection calls that perform overt
communication. Our analysis employs a class hierarchy analysis
(CHA)~\cite{Dean1995} to build a call graph with refinement achieved
by intra-procedural data-flow analysis, as discussed later.  After much
experimentation with higher precision, though brittle, points-to
analysis techniques, this analysis combination gave us the best
performance for the classification task.  Our analysis is implemented
in the Soot Analysis Framework~\cite{Vallee-Rai2000} and utilizes the
Android API model provided by
DroidSafe~\cite{Gordon:Kim:Perkins:Gilham:Nguyen:Rinard:NDSS15}. The
presentation of the analysis below assumes the application is
represented in the Jimple intermediate language~\cite{Vallee-Rai2000}.

To compute a call graph, we augment the application code with the
DroidSafe Android Device Implementation
(ADI)~\cite{Gordon:Kim:Perkins:Gilham:Nguyen:Rinard:NDSS15}.  The ADI
is a Java-based model of the Android runtime and API. It attempts to
present full runtime semantics for commonly-used classes of the
runtime and API.  Our call graph construction begins at each
application method that could be a target of the Android runtime, i.e., 
the event handler methods of the application.  Call graph
construction does not traverse into Android API methods. 
As such, the call
graph is actually a forest of graphs, each rooted at an event
handler of the application, and each graph includes only
application-defined methods.  However, we found it necessary to
account for API calls that directly call back into the application,
e.g., \lstinline!Thread.start()!. We achieve that by replacing the API call with a
direct call to the application method(s) that the API call could
invoke, e.g., \lstinline!Thread.run()!.

%% For example, if an application method, $m$ calls
%% \lstinline!Thread.start()! on a receiver that is an application class,
%% $t$, we found it necessary to add the edge to the call graph ($m$,
%% $t$.\lstinline!run()!).  This includes the started thread $t$ in
%% failure handler if $m$ is encountered.

%% To achieve this in general, we add to the call graph edges of the type
%% ($m$, $n$) where there is an edge ($m$, \lstinline!api-method!), the
%% call of \lstinline!api-method! is passed a value that is a reference
%% to an application class, and \lstinline!api-method! calls method $n$
%% on the passed application class value.  This strategy adds to our
%% call graph the edges for the \lstinline!Thread.start()! to the
%% \lstinline!Thread.run()! discussed above.

The call graph is augmented to account for reflected
method calls using the following policy.  When a
reflected call is found, we add edges to the graph that target all
methods of the same package domain as the caller (e.g.,
\lstinline!com.google!, \lstinline!com.facebook!).  The edges are pruned by the
following strategy: if the number of arguments and argument types to
the call can be determined using a def-use analysis~\cite{Aho2006},
then we limit the edges to only targets that have the same number and
types of arguments.  This strategy works well for us in
practice and aggressively accounts for reflection semantics.



\lstDeleteShortInline+
\lstMakeShortInline[basicstyle=\scriptsize\ttfamily,keywordstyle=\color{DarkPurple},breaklines=true]+

\begin{table}[t]
\small
\renewcommand*{\arraystretch}{1.3}
\caption{Considered UI Elements.}
\label{tbl:ui}
\centering
\tabcolsep=1.5pt
\resizebox{\columnwidth}{!}{%
\begin{tabular}{|l|P{4.7cm}|P{5.7cm}|}
\hline
& \textbf{Class or Interface} & \textbf{Methods} \\
\hline
1. & android.app.Dialog                                 					& setContentView \\
2. & android.support.v7.app.\newline{}ActionBarActivityDelegate  			& setContentView \\
3. & android.view.View                                  					& onLayout, layout, onDraw, onAttachedToWindow \\
4. & android.view.ViewGroup                             					& addView, addFocusables, addTouchables, addChildrenForAccessibility \\
5. & android.view.ViewManager                           					& addView, updateViewLayout \\
6. & android.view.WindowManagerImpl.\newline{}CompatModeWrapper  	        & addView \\
7. & android.webkit.WebView                             					& loadData, loadDataWithBaseURL, loadUrl \\
8. & android.widget.TextView       											& append, setText \\
9. & android.widget.Toast        											& makeText \\
\hline
\end{tabular}
}%resizebox
\end{table}

\lstDeleteShortInline+

\lstMakeShortInline[basicstyle=\scriptsize\ttfamily,keywordstyle=\color{DarkPurple},breaklines=false]+



\subsection{Failure Handler Analysis}
\label{sec:failure-handling}

Failure analysis determines if a connection call modifies the UI or
could exit the application on exception.  We organize the static
failure-handling analysis as a recursive traversal on the call graph.
An iterator over all application statements calls the analysis
separately for the combination of each statement in the application
that could target a connection call and an exception that indicates
communication failure.

Conceptually, the failure handling of $s$ on $e$ is defined as a slice
of instructions.  In the slice are all statements that are reachable
from all \lstinline!catch! blocks that can handle $e$.  Then, for each
reachable statement in the slice that could throw an exception of type
$f$, we recursively add to the slice all statements that are reachable
from all \lstinline!catch!  blocks that could dynamically handle $f$.
Next, we recursively process those newly added statements that could
throw an exception.  In the processing, Android API methods are not
considered, and searching for a handler does not propagate into the
Android runtime environment.

%resize line numbers
\algrenewcommand\alglinenumber[1]{\scriptsize #1:}
\begin{figure}
\scriptsize
\begin{algorithmic}[1]
\algblockdefx[HEADER]{inputdef}{outputdef}[1]%
{\textbf{Input}: #1}[1]{\textbf{Output}: #1}
\Procedure{FindCatches}{meth, stmt, ex, visiting, stack, cg}

\IIf {(stmt, ex) $\in$ visiting \textbf{or} stmt $\in$ overt}
\textbf{return} \EndIIf

\State visiting $\gets$ visiting $\cup$ (stmt, ex)

\State catchBlockStart $\gets$ \textsc{FindCompatCatch}(meth, stmt, ex)
\label{alg:find-catch}
\If {catchBlockStart $=$ \textbf{null} }
\label{alg:no-local-hander-start}
\If {\textsc{IsEventHandler}(meth)}
\label{alg:event-handler-line}
\State overt $\gets$ overt $\cup$ stmt
\State \textbf{return}
\EndIf
\label{alg:event-handler-end}

\For { (predStmt, predMeth) $\in$ \textsc{GetPreds}(cg, meth)}
\label{alg:no-local-hander-start}
\If {stack $\neq \emptyset$ \textbf{and} (predStmt, predMeth) $\neq$
  \textsc{Peek}(stack)} 
\label{alg:stack-test-line}
\State \textbf{continue}
\EndIf 

\State newStack $\gets$ stack
\State \textsc{Pop}(newStack)
\label{alg:stack-pop-line}
\State \textsc{FindCatches}(predMeth, predStmt, ex, 
\State         \hspace{2cm}visiting, newStack, cg)

\If {predStmt $\in$ overt}
\State overt $\gets$ overt $\cup$ stmt
\State \textbf{return}
\EndIf
\label{alg:no-local-hander-end}
\EndFor

\Else 
\State catchStmts $\gets$ \textsc{GetCatchStmts}(catchBlockStart, meth)
\label{findcatches-analyze-call-start}
\State \textsc{AnalyzeHandler}(meth, stmt, catchStmts,
\State \hspace{2cm}visiting, $\emptyset$, stack, cg)
\label{findcatches-analyze-call-end}
\EndIf

\EndProcedure
\end{algorithmic}
\vspace{-0.05in}
\caption{Find \lstinline!catch! blocks for exception thrown at statement.}\label{alg:findcatches}
\vspace{-0.1in}
\end{figure}

The analysis starts with the \textsc{FindCatches} procedure listed in
Fig.~\ref{alg:findcatches}.  It employs a set of helper procedures;
the non-obvious ones are listed in Fig. 4, and the full list is available
in~\cite{helper-funs-web-page}. \textsc{FindCatches} is called
separately for each connection statement and exception pair, stmt and
ex, respectively.  The arguments to the procedure also include stmt's
enclosing method (meth); the set of currently visiting statements and
exception pairs (visiting), initially empty; the call stack built when
searching forward on a path of the call graph (stack), initially
empty; and the pre-calculated call graph (cg).

For stmt and ex, the procedure first consults stmt's containing method
to find an appropriate \lstinline!catch! (line~\ref{alg:find-catch}).
If ex is not caught locally, and meth is an event handler (called from
the Android runtime), then we conservatively calculate that stmt could
cause application exit, and it is added to the overt set
(lines~\ref{alg:event-handler-line}-\ref{alg:event-handler-end}).
Otherwise, the analysis recursively visits all direct caller methods
to find \lstinline!catch! blocks that trap the call graph edge (lines
\ref{alg:no-local-hander-start}-\ref{alg:no-local-hander-end}), as
discussed later.

\begin{figure}[t]
\begin{algorithmic}[1]
\scriptsize
\algblockdefx[HEADER]{inputdef}{outputdef}[1]%
{\textbf{Input}: #1}[1]{\textbf{Output}: #1}
\Procedure{AnalyzeHandler}{meth, exceptStmt, stmts, 
  visiting, handledStmts, stack, cg}

\IIf {stmts $\in$ handledStmts} \textbf{return} \EndIIf

\State handledStmts $\gets$ handledStmts $\cup$ stmts

\For {each stmt $\in$ stmts}
\If {\textsc{HasInvoke}(stmt)}
\For {(succStmt, succMeth) $\in$ \textsc{GetSuccs}(cg, stmt)}

\If {\textsc{IsUIMethod}(succMeth)}
\label{alg:ui-method-found}
\State overt $\gets$ overt $\cup$ exceptStmt
\State \textbf{return}

\ElsIf {\textsc{IsNativeMethod}(succMeth)}
\label{alg:native-method-line}
\For {nativeEx $\in$ \textsc{GetThrowsExceptions}(succMethd)}
\State \textsc{FindCatches}(meth, stmt, nativeEx, visiting, stack, cg)
\EndFor

\Else
\label{alg:ah-app-method-call-start}
\State newStack $\gets$ stack
\State \textsc{Push}(newStack, (succStmt, succMeth))
\label{alg-push-stack-lines}
\State succStmts $\gets$ \textsc{GetBodyStmts}(succMeth)
\State \textsc{AnalyzeHandler}(succMeth, exceptStmt, succStmts,
\State \hspace{2cm}visiting, handledStmts, newStack, cg)
\label{alg:ah-app-method-call-end}
\EndIf 

\EndFor

\ElsIf {\textsc{IsThrowStmt}(stmt)}

\State rethrownTypes = $\emptyset$
\label{alg:found-throws-start}
\For {defStmt $\in$ \textsc{GetLocalDefs}(\textsc{GetOp}(stmt))}

\If {\textsc{IsAlloc}(defStmt)}
\State rethrownTypes $\gets$ rethrownTypes $\cup$ \textsc{GetType}(defStmt)
\ElsIf {\textsc{IsCaughtExceptionStmt}(defStmt)}
\State rethrownTypes $\gets$ rethrownTypes $\cup$ 
\State \hspace{0.5cm} \textsc{GetPossibleThrownTypes}(meth, defStmt)
\label{alg:getpossiblethrowntypes}
\Else
\label{alg:unknown-last-def-throw}
\State overt $\gets$ overt $\cup$ exceptStmt
\State \textbf{return}
\EndIf

\EndFor \label{alg:calc-rethrow-types-end}

\For {rethrownType $\in$ rethrownTypes}
\label{alg:findcatches-rethrown-line}
\State \textsc{FindCatches}(meth, stmt, rethrownType, visiting, stack,
cg)

\If {stmt $\in$ overt}
\label{alg:progagate-line}
\State overt $\gets$ overt $\cup$ exceptStmt
\State \textbf{return}
\EndIf
\EndFor
\label{alg:found-throws-end}
\EndIf

\EndFor

\EndProcedure
\end{algorithmic}
\caption{Analyze reachable statements during failure handling.}\label{alg:analyzehandler}
\vspace{-0.1in}
\end{figure}

If a compatible handler is found locally (lines
\ref{findcatches-analyze-call-start}-\ref{findcatches-analyze-call-end}),
the analysis calls the procedure \textsc{AnalyzeHandler}
(Fig.~\ref{alg:analyzehandler}) on the statements of the compatible
block.  This procedure analyzes the reachable statements of the
handler. If a call that could target a UI method is encountered, then
the statement that began the handler analysis is considered overt
since the failure handling affects the user interface
(lines 7-9 in Fig.~\ref{alg:analyzehandler}). %~\ref{alg:ui-method-found}).  
If the analysis finds a call to a
native method, we assume that the method will throw all exceptions it
is defined to throw, and the handler analysis spawns a
\textsc{FindCatches} instance for each exception declared throws
(lines 10-13). %~\ref{alg:native-method-line}). 
When the analysis finds a call to
an application method, it pushes the current statement and method onto
the stack and recursively calls itself for the new method to analyze
the new method's statements (lines
\ref{alg:ah-app-method-call-start}-20). %\ref{alg:ah-app-method-call-end}).

If the analysis finds a \lstinline!throw!  statement, it %the handler analysis
spawns a new \textsc{FindCatches} analysis to find all the possible
handlers of each rethrown exception (lines 22-42).
%\ref{alg:found-throws-start}-\ref{alg:found-throws-end}).
%During handler analysis, if a \lstinline!throw! statement is
%encountered, the analysis needs to
%To determine the possible type(s) of the thrown value, and then start a new
%search for the handler(s) of the type(s).  The \textsc{AnalyzeHandler} procedure
Towards this end, it first calculates local def-use chains to obtain the types of the
exception.  In lines~\ref{alg:found-throws-start}-\ref{alg:calc-rethrow-types-end},
the analysis considers all local reaching defs of the thrown value.  If an allocation statement
reaches the \lstinline!throw!, then the allocated type is added to the
set of possible types of the rethrown exception. If a caught exception
reference, $c$, reaches
the \lstinline!throw! statement, then the \lstinline!try! block
associated with \lstinline!catch! block of $c$ is analyzed for all
checked exceptions that could be thrown.  This is performed in
the \textsc{GetPossibleThrownTypes} call in line
\ref{alg:getpossiblethrowntypes} of \textsc{AnalyzeHandler}.  If only
allocations and caught exception statements reach the thrown value,
then the handler analysis spawns a new \textsc{FindCatches} instance
to analyze the failure handling.


%% The analysis considers the reachable statements
%% inter-procedurally and flow-insens\-itively.  Handler analysis
%% searches for: (1) calls to application methods, (2) \lstinline!throw!
%% statements (3) calls to native methods, and (4) possible calls to UI
%% methods. 

%% \textsc{FindCatches} and \textsc{AnalyzeHandler} maintain a set of
%% statement and exception pairs, overt, that records pairs that are
%% calculated as overt.  After all connection call statement and
%% exception pairs are analyzed, pairs not in the overt set are
%% considered covert.

A stack of pairs of method call statement and target method is
maintained.  The analysis uses the stack to focus the handler search
in \textsc{FindCatches} after a method call has been performed by a
handler further up the stack (lines
\ref{alg:stack-test-line}-\ref{alg:stack-pop-line} in Fig.~\ref{alg:findcatches}). When we initiate the analysis for a
connection call, the stack is empty and the analysis in
\textsc{FindCatches} has to search all possible stacks (predecessor of
the containing method) for handlers of the connection statement's
exception.  However, once a handler is found, and the handler calls a
sequence of methods that ends in a possible rethrown exception, the
sequence of methods defines the only stack that should be searched for
a handler of the rethrown exception (line \ref{alg:stack-test-line}).

The stack is pushed in line \ref{alg-push-stack-lines} of
\textsc{AnalyzeHandler} for each method call of a reachable handler
code.  During the handler search of the execution stack in
\textsc{FindCatches}, the stack is consulted to guide the search in
line \ref{alg:stack-test-line}, only visiting the edge at the head
of the stack. The stack is popped when visiting a caller method of the
current method in \textsc{FindCatches} line~\ref{alg:stack-pop-line}.



%% If any other type of statement is a definition that reaches the thrown
%% value, then the analysis cannot determine the exception type and the
%% connection call (or rethrown exception statement) is considered
%% overt (line \ref{alg:unknown-last-def-throw}). 

%% \begin{figure}[t]
%% \begin{algorithmic}[1]
%% \scriptsize
%% \Procedure{GetPossibleThrownTypes}{meth,stmt}
%% \State thrownTypes = $[]$
%% \State tryStmts = \textsc{GetTryBlock}(meth,stmt)

%% \For {tryStmt $\in$ tryStmts}

%% \If {\textsc{IsThrowStmt}(tryStmt)}

%% \For {defStmt $\in$ \textsc{GetLocalDefs}(\textsc{GetOp}(stmt))}
%% \If {\textsc{IsAlloc}(defStmt)}
%% \State rethrownTypes $\gets$ rethrownTypes $\cup$ 
%% \State \hspace{1cm} \textsc{GetAllocType}(defStmt)
%% \ElsIf {\textsc{IsCaughtExceptionStmt}(defStmt)}
%% \State reThrownTypes $\gets$ rethrownTypes $\cup$
%% \State \hspace{1cm} \textsc{GetPossibleThrownTypes}(meth, defStmt)
%% \Else 
%% \State \textbf{return} $\emptyset$
%% \label{alg:getpossiblethrown-return-null}
%% \EndIf
%% \EndFor

%% \ElsIf {\textsc{HasInvoke}(tryStmt)}

%% \For {(succStmt, succMeth) $\in$ \textsc{GetSuccs}(cg, tryStmt)}
%% \State thrownTypes $\gets$ thrownTypes $\cup$ 
%% \State \hspace{1cm}\textsc{GetThrowsExceptions}(succMeth)
%% \EndFor

%% \EndIf

%% \EndFor

%% \State \textbf{return} thrownTypes

%% \EndProcedure
%% \end{algorithmic}
%% \caption{Calculate exception types caught at statement.\label{alg:GetPossibleThrownTypes}}
%%   \vspace{-0.1in}
%% \end{figure}


\lstDeleteShortInline+

\lstMakeShortInline[basicstyle=\scriptsize\ttfamily,keywordstyle=\color{DarkPurple},breaklines=true]+

\begin{figure}[t]
\scriptsize
\renewcommand*{\arraystretch}{1.3}
\begin{tabular}{|p{3.3in}|}
\hline 

\textsc{FindCompatCatch}($meth$,$stmt$,$ex$): Return the first
  statement of the +catch+ block that will handle an
  exception of type $ex$ thrown at statement stmt in method meth.
%% \\
%% \hline
%% \textsc{IsEventHandler}($meth$): Return true if method $meth$
%%   overrides a method defined in the Android API.  This method
%%   over-approximates the methods that can be called by the Android
%%   runtime to handle events.
\\
\hline
\textsc{GetCatchStmts}($stmt$,$meth$): Given the start of a
+catch+ block defined in the trap table of method $meth$,
return all statements that were defined in the source code for the
+catch+ block of $stmt$.  This method calculates an
over-estimation of +catch+ block extents, e.g., it includes
+finally+ blocks.
\\
\hline
\textsc{GetPossibleThrownTypes}($meth$,$stmt$): Calculate the
possible exception types caught at the catch block that begins with
$stmt$ of $meth$. The statements of the +try+ block that associates with the +catch+ block
that encloses $stmt$:  (1) For a call statement, the
procedure adds to the return list all exception types declared throws
by all methods that the call can target,  (2) For a +throw+
statement, the reaching definitions of the thrown value are
calculated.  If the reaching definition is an allocation, then add to
the return list the type of the allocation.  If the reaching
definition is a caught exception statement, then
\textsc{GetPossibleThrownExceptions} recursively calls itself to find
the nesting try block statements and continue the calculation.  If a
definition of any other statement type can reach the thrown value,
then return null to denote that it cannot calculate the
thrown.  
\\
\hline
%% Used in the definition of GetPossibleThrownTypes
%% \textsc{GetTryBlock}($meth$,$stmt$): Given a statement $stmt$
%%   that begins a +catch+ block in method $meth$, return the
%%   list of statement of try block associated with the enclosing
%%   +catch+ block of $stmt$.
%% \\
%% \hline
\end{tabular}
\caption{\label{fig:helper-funs}Failure handling analysis helper functions.}
\vspace{-0.1in}
\end{figure}

\lstDeleteShortInline+



%% \begin{description}[leftmargin=0cm,listparindent=0pt,itemindent=0cm]

%% \item \textsc{FindCompatCatch}($meth$,$stmt$,$ex$): Return the first
%%   statement of the \lstinline!catch! block that will handle an
%%   exception of type $ex$ thrown at statement stmt in method meth.

%% \item \textsc{IsEventHandler}($meth$): Return true if method $meth$
%%   overrides a method defined in the Android API.  This method
%%   over-approximates the methods that can be called by the Android
%%   runtime to handle events.

%% \item \textsc{GetCatchStmts}($stmt$,$meth$): Given the start of a
%%   \lstinline!catch! block defined in the trap table of method $meth$,
%%   return all statements that were defined in the source code for the
%%   \lstinline!catch! block of $stmt$.  Since the dex bytecode does not
%%   provide the ending statement of traps, we need to calculate the
%%   extent of the catch block.  \textsc{GetCatchStmts} takes advantage
%%   of the property that Java compilers do not produce code that jumps
%%   from outside of the catch block into the middle of a catch block.
%%   So to calculate the \lstinline!catch! block's extent,
%%   \textsc{GetCatchStmts} (1) produces a control flow graph (CFG) for
%%   $meth$, (2) colors all statements reachable from $stmt$ in the CFG,
%%   (3) for each statement, $c$ of $meth$, if all predecessors of $c$ in
%%   the CFG are colored then $c$ is included in the set of statements
%%   that are returned ($stmt$ is also included in the return set). This
%%   method calculates an over-estimation of \lstinline!catch! block
%%   extents, e.g., it includes \lstinline!finally! blocks.

%% \item \textsc{GetTryBlock}($meth$,$stmt$): Given a statement $stmt$
%%   that begins a \lstinline!catch! block in method $meth$, return the
%%   list of statement of try block associated with the enclosing
%%   \lstinline!catch! block of $stmt$.

%% \end{description}
 
\subsection{Success Analysis}

For connection statement stmt, if the failure handling analysis
concludes that no UI call could be invoked during failure handling and
all stacks handle stmt's exception, then the analysis continues with
the success paths of stmt. The success analysis determines if there is
any UI modification after the connection succeeds but before control
returns to the Android runtime environment and before control merges
back to the failure handling paths.

Here we summarize the success analysis of connection call stmt enclosed
in method meth by presenting a high-level description of two conceptual
phases:

\subsubsection{Success Forward Analysis} Code reachable from the
  statement immediately after stmt is searched for a connection call.
  This is accomplished by traversing all paths in the interprocedural
  control flow graph (CFG) starting at stmt and following method invoke
  expressions via the call graph.  We follow both normal and
  exceptional control paths when analyzing the CFG.

\subsubsection{Success Backward Analysis} For all methods, caller, such
  that there is a path from caller to meth in the call graph and caller was
  searched during the failure handling analysis, examine all
  statements of caller for calls that affect the user interface.  This
  has the effect of traversing the call graph backwards from meth at
  stmt for UI calls, stopping at event handlers (that are called by the
  Android API) and, on a path, stopping once the exception of stmt is
  handled.

If the success forward and success backward analysis do not find any
calls that could affect the user interface, then the connection call
is classified as covert.

\subsection{Design Discussion}

The intuition for our static analysis is that overt connection calls
{\it always} affect the user interface either on success or
failure. Furthermore, the effect is {\it direct} on success.  An
execution trace that handles a single runtime event naturally denotes
a unified block of processing. As such, our analysis reasons about
event handlers to bound the success search to represent the direct
processing of a connection call.

Another insight gleaned from the study in Section~\ref{sec:study} is
that direct processing associated with a successful invocation of
connection call stmt in method meth often ends when the successful control
flow merges back with the exceptional control flow. This is the
motivation for why the success backward analysis only examines methods
that were analyzed by the failure analysis.  

